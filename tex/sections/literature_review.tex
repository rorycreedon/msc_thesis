\chapter{Literature Review} \label{chapter:lit_review}

\section{Algorithmic Recourse}

\subsection{Motivation}
Description of what algorithmic recourse is and why it is important - use of automatic decision making, GDPR \citep{voigtEUGeneralData2017}. Mention psychological factors causing humans to prefer recourse to explanations (\textbf{to find paper(s)}: was mentioned by Ruth Byrne in \href{https://icml.cc/virtual/2021/11705}{ICML panel session}, from 25 minutes onwards).


\subsection{Problem Set-up}
\begin{itemize}
	\item Description of the original set-up and problem - i.e., 	
	\begin{align} \label{eq:recourse_setup}
		\mathbf{x}^f = & \argmin_{\mathbf{x}' \in \mathbf{\mathcal{X}}} c(\mathbf{x},\mathbf{x}') \\
		\text{s.t. } & h(\mathbf{x}') = 1, \nonumber \\ 
		& c(\mathbf{x},\mathbf{x}') \leq B \nonumber
	\end{align} 
	
	\item Description of the causal recourse set-up and problem \citep{karimiAlgorithmicRecourseCounterfactual2021}
	\item Cost and distance functions, actionability of features
\end{itemize}


\subsection{Recourse methods}
Run through methods mentioned in survey paper \citep{karimiSurveyAlgorithmicRecourse2022} and also those implemented in \href{https://carla-counterfactual-and-recourse-library.readthedocs.io/en/latest/recourse.html}{\texttt{CARLA}}.





\section{Strategic Classification}

\subsection{Standard Strategic Classification}


\begin{itemize}
	\item Begin with \textcite{hardtStrategicClassification2016} and explain the set-up as a Stackelberg game with an example.
	
	\item Algorithms proposed for this task include  \textcite{levanonStrategicClassificationMade2021}, \textcite{chenLearningStrategyAwareLinear2020} and \textcite{ahmadiClassificationStrategicAgents2022}. 
	
	\item Mention extensions such as:
	
	\item Where the cost function is completely unknown to the lender \citep{dongStrategicClassificationRevealed2018}
	
	\item Where the response of lenders to the classifier is noisy \citep{jagadeesanAlternativeMicrofoundationsStrategic2021}.
	
	\item Where borrowers do not know the decision rule \citep{ghalmeStrategicClassificationDark2021, bechavodInformationDiscrepancyStrategic2022}.
	
	\item Where the incentives of lender and borrower align (e.g., recommender systems) \citep{levanonGeneralizedStrategicClassification2022}.
	
	\item Where the cost functions are linked by graphs for the borrowers \citep{eilatStrategicClassificationGraph2023}.
	
	\item Where the borrowers act first \citep{nairStrategicRepresentation2022}.
	
	\item Where the borrowers and lenders update at different rates \citep{zrnicWhoLeadsWho2021}.
\end{itemize}

\subsection{Causal Strategic Classification}
A review of the \textit{causal} strategic classification literature, which focuses more on causal identification of features which are strategically manipulated (without causing an improvement in underlying credit `worthiness') and features which causally affect credit `worthiness'.



\section{Revealed Preferences} \label{section:revealed_pref_lit}
A brief primer on axioms of revealed preferences, and on the literature of \textit{learning from revealed preferences}. To briefly discuss:

\begin{itemize}
	\item Original paper by \textcite{beigmanLearningRevealedPreference2006}, where principal issues a list of prices and the agent purchases different quantities of each good. Over time, the principal learns from the different purchase amounts (which are the revealed preferences).
	\item When prices are of goods and budget of the agent are drawn from an unknown distribution \citep{zadimoghaddamEfficientlyLearningRevealed2012, balcanLearningEconomicParameters2014}.
	\item Where the principal is maximising profit \citep{aminOnlineLearningProfit2015, rothWatchLearnOptimizing2016}.
	\item Move onto a more detailed discussion of \textcite{dongStrategicClassificationRevealed2018}.
\end{itemize}


\section{Pairwise Metric Learning}
\begin{itemize}
	\item Start with an introduction of what pairwise metric learning is and key papers.
	\item Move onto specific proposed adaptation/simplification of the learning algorithm proposed in \textcite{canalOneAllSimultaneous2022}.
\end{itemize}


\section{Canonical Datasets}
The canonical datasets used in the algorithmic recourse and strategic classification literature include:

\begin{itemize}
	\item \href{https://archive.ics.uci.edu/dataset/2/adult}{\texttt{Adult}} - dataset to predict whether someone earns over \$50,000 or more.
	\item \href{https://archive.ics.uci.edu/dataset/144/statlog+german+credit+data}{\texttt{German Credit}} - dataset identifies people as either good or bad credit risks.
	\item \href{https://community.fico.com/s/explainable-machine-learning-challenge}{\texttt{FICO-HELOC}} - dataset of HELOC applications, where applicants have applied for a credit line between \$5,000 and \$150,000. Outcome variable is whether they are a good or bad credit risk.
	\item \href{https://www.kaggle.com/datasets/shebrahimi/financial-distress}{\texttt{Finance}} - dataset to predict financial distress for a number of companies. There are several over different time periods for each company.
\end{itemize}



