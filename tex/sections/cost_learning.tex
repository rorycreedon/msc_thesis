\chapter{Cost Learning} \label{chapter:cost_learning}

In the introduction, three key issues were highlighted with existing cost function in the algorithmic recourse. The first issue, the use of sequential interventions, were covered in chapter \ref{chapter:causal_recourse}. The second and third issues, taking into account preferences over feature mutability and specification of the underlying SCM, are covered in this chapter. We outline a novel methodology to simultaneously learn user preferences over feature mutability and specification of the SCM from the users' \textit{revealed preferences}. \\


User preferences over feature mutability, or how easy it is for individuals to change different features, is important to take into account in our cost function. Without taking into account individual preferences over feature mutability, an individual with very low discretionary spending and highly irreducible costs (e.g., childcare) and limited opportunities for a salary increase may be asked to significantly increase their savings in order to be approved for a loan, instead of providing recourse in more more mutable features, such as improving credit score and number of loans successfully paid back. \\

The SCM used to calculate causal recourse is also an important to take into account. As outlined in section \ref{section:causal_recourse_motivation}, in order to correctly measure the cost of changing features from $\mathbf{x}$ to $\mathbf{x}'$, we need to take into account the causal effects of changing one variable on the others. However, if we misspecify the SCM, then we inaccurately measure the causal of effect of changing one variables on other variables, which will lead to sub-optimal recourse provided to negatively classified individuals.

\section{Learning from Revealed Preferences}

Revealed preference theory \citep{samuelsonNotePureTheory1938, samuelsonConsumptionTheoryTerms1948} is an economic theory that states that if a consumer if offered a bundle of goods $x$ and a bundle of goods $y$ both within a budget set $B$ and the consumer chooses $x$ over $y$, then $x$ is preferred to $y$. From the consumer's decision to purchase bundle $x$ over bundle $y$, the consumer \textit{reveals} their (normative) \textit{preferences} over $x$ and $y$.

We can use this idea to learn individuals' revealed preferences over feature mutability. Whilst humans are not able to describe their own preferences mathematically, they are able to express their preferences when given limited options to choose from. We propose to ask negatively classified individual a series of questions. In each question, individuals are asked to compare two sets of ordered actions $A^1$ and $A^2$ and responds with whichever one is least costly. If $A^1$ is preferred to $A^2$. We can use this information to parameterise a cost function which takes into account user preference.

A demonstration of one of the pairwise comparisons that users are asked to evaluate are shown in Figure \ref{fig:comparison_ui}. We envision that users are asked to evaluate these comparisons in an online system, where they must answer the pairwise comparisons in order for recourse to be generated and presented to the users.

\begin{figure}[!htb]
	\centering
	\begin{tabular}{l|l}
		\hline
		\textbf{Action Sequence 1} & \textbf{Action Sequence 2} \\
		\hline
		\textsc{1. Salary} $\to$ £70,000 & \textsc{1. Savings} $\to$ £45,000 \\
		\multicolumn{1}{c|}{\textit{then}} & \multicolumn{1}{c}{\textit{then}}\\
		\textsc{2. Savings} $\to$ £52,000 & \textsc{2. Debt} $\to$ £5,000 \\
		\multicolumn{1}{c|}{\textit{then}} & \multicolumn{1}{c}{\textit{then}}\\
		\textsc{3. Debt} $\to$ £7,000 & \textsc{3. Salary} $\to$ £65,000 \\ 
		\hline
	\end{tabular}
	
	\vspace{1.5em} % Add some vertical space between the table and the text.
	\parbox{\linewidth}{
		\centering
		\textit{Question for user:}\\
		Which of the above sequences of actions are easier to complete? \\
		$\Box$ Action Sequence 1 \\
		$\Box$ Action Sequence 2
	}
	
	\caption{Hypothetical pairwise comparison a user is asked to respond to.}
	\label{fig:comparison_ui}
\end{figure}

In order to effectively learn from the responses of the negatively classified individuals, we make an assumption - that individuals can evaluate their actions using the true SCM. Whilst on the surface, this may appear to be a very strong assumption, it is not unreasonable to think that, in certain settings such as credit scoring, an individual could evaluate the effect of a change in one variable, for example income from £50,000 to £70,000 on other variables such as savings and debt. This is because an individual would be able to estimate how much of the additional income they would be able to save, how much they could use to pay down any existing debt, and how much would likely be allocated to discretionary spending. We do not assume complete knowledge of the SCM - we do not assume that individuals are able to evaluate causal effects for any individual other than themselves. We assume that individuals have 'local' knowledge of the SCM. Later in this chapter, we relax this assumption slightly to assume that individuals have noisy, local knowledge of the SCM.

\section{Cost Learning Formulation} \label{section:cost_learning_formulation}

We now formulate the cost learning problem, where there are $N$ individuals who are each presented with $K$ pairwise comparisons. In each comparison, each individual compares two sets of ordered actions $(A^1, A^2)$. We record the response of the $n^{\text{th}}$ individual to their $k^{\text{th}}$ comparison as $y_{kn}$, which is defined below, where $\mathbf{x}_n$ represents the $n^{\text{th}}$ individuals' original features.



\begin{equation} \label{eq:paired_response}
	y_{kn} = \begin{cases}
		-1 & \text{if } A^1_{kn} \text{ preferred to } A^2_{kn} \\
		+1 & \text{if } A^2_{kn} \text{ preferred to } A^1_{kn}
	\end{cases}
\end{equation}



\subsection{Perfect Knowledge of the Structural Causal Model}

Let us first consider the case where both the deployer of the model and the users have perfect knowledge of the underlying SCM $\mathcal{M}$. In this case, the individual $k$ has a cost function $\cost(\mathbf{x}, A|\beta_k, \mathcal{F})$, parameterised by $\mathcal{F}$, the corresponding structural equations of the SCM $\mathcal{M}$ and $\beta_k$, a vector which represents how easily mutable each feature is. As the model deployer has perfect knowledge of the SCM, they do not need to learn (or approximate) $\mathcal{F}$. The model deployer's task become to learn a $\beta_i$ for each individual which represents each users' preferences (of how mutable each feature is) that explains as many of the individuals' responses as possible. We denote the model deployers' predicted response as $\hat{y}_{kn}$, which is defined below, where $\lambda$ is a hyperparameter regularising for `confidence' of the predictions. As true values $y_{kn} \in \{-1,1\}$, the hyperbolic tangent function is used to squash the predicted values $\hat{y}_{kn}$ into $[-1,1]$.

\begin{equation} \label{eq:yhat_relaxation}
	\hat{y}_{kn} = \tanh \Bigg(\lambda \Big(\cost(\mathbf{x}_n, A^1_{kn}|\beta_k, \mathcal{F}) - \cost(\mathbf{x}_n, A^2_{kn}|\beta_k, \mathcal{F})\Big)\Bigg)
\end{equation}

The model deployer's task can be achieved through the below objective, where there are $K$ individuals and $N$ pairwise comparisons and $\ell(y, \hat{y}) = \max[0, 1-y\hat{y}]$ represents the hinge loss.

\begin{equation}
	\beta^* = \argmin_\beta \frac{1}{KN} \sum_{k=1}^K \sum_{n=1}^N \ell (y_{kn}, \hat{y}_{kn}) + \underbrace{\lambda ||\beta_k||_2}_{\text{L2 regularisation}}
\end{equation}

This is an unconstrained optimised problem that can be optimised using gradient descent. In order to avoid overfitting to responses of the sample of the negatively classified individuals who answer the pairwise comparisons, L2 regularisation is also added to the objective function.

\subsection{Linear Approximation of the Structural Causal Model}

Now we consider the (more realistic) case where the model deployer does not have perfect knowledge of the structural causal model. In this case, the model deployer learns both the user preferences $\beta_k$ as well as an approximation of the structural equations $\mathcal{F}$.\\


In this case, model deployer knows neither the form of the structural equations $\mathcal{F}$ nor the parameters of the equations. For example, the true structural equations could be linear, as shown below for a three variable SCM.

\begin{align} \label{eq:linear_scm_example}
	x_1 & = u_1 & u_1 \sim N(\mu_1,\sigma_1^2) \\ \nonumber
	x_2 & = ax_1 + u_2 & u_2 \sim N(\mu_2,\sigma_2^2) \\ \nonumber
	x_3 & = bx_1 + cx_2 + u_3 & u_3 \sim N(\mu_3,\sigma_3^2) \\ \nonumber
\end{align}

The structural equations could also be a more complex non-linear set of equations. An example is shown below.

\begin{align}
	x_1 & = u_1 & u_1 \sim N(\mu_1,\sigma_1^2) \\ \nonumber
	x_2 & = ax_1 + bx_1^2 + u_2 & u_2 \sim N(\mu_2,\sigma_2^2) \\ \nonumber
	x_3 & = \frac{cx_1}{1 + \exp(dx_2)} + u_3 & u_3 \sim N(\mu_3,\sigma_3^2) \\ \nonumber
\end{align}

The model deployer knows neither the form of the equations nor the parameters (e.g, $a,b,c,d$ in the above example). As a solution to this problem, we learn a \textit{linear approximation} of the SCM\footnote{A linear approximation of the SCM is used in this thesis largely for its simplicity. However a more flexible specification of the approximation of the SCM, such as kernel ridge regression, could be used in future works.}.\\

In the linear approximation of the SCM, each variable is modelled as a linear combination of the other variable. We can represent the parameters of the linear approximation of the SCM as a square matrix $W$, where each element of the matrix represents $\frac{\partial x_j}{\partial x_i}$. As each variable $x_j$ is a linear combination of the other variables $x_{\neg j}$, $\frac{\partial x_j}{\partial x_i}$ will be a scalar (and will be 0 if $x_j$ has no causal effect on $x_i$). For the linear SCM described in equation \ref{eq:linear_scm_example}, the parameters $W$ can be described as shown below. It can be thought of as a weighted adjacency matrix, where the weights are the marginal effects of $x_j$ on $x_i$.

\begin{equation}
	W = 
	\begin{bmatrix}
		1 & a & ab+c \\
		0 & 1 & c \\
		0 & 0 & 1
	\end{bmatrix}
\end{equation}

With a linear approximation of the SCM, the model deployer's task is now to learn the parameters $W$ and user preferences $\beta_k$. We add $W$ in to the objective function as follows.

\begin{equation}
	\beta^*, W^* = \argmin_{\beta, W} \frac{1}{KN} \sum_{k=1}^K \sum_{n=1}^N \ell (y_{kn}, \hat{y}_{kn}) + \underbrace{\lambda_1 ||\beta_k||_2 + \lambda_2||W||_2}_{\text{L2 regularisation}}
\end{equation}

We additionally restrict $W$ such that the diagonal is fixed at 1 - meaning that an intervention of size $\delta_i$ on variable $i$ has a causal effect of $\delta_i$. For example, if we increased salary from £30,000 to £35,000 ($\delta_i=£5,000$), then income after the intervention is £35,000. It does not lead to downstream (causal) effects that result in salary changing to any value other than £35,000 ($x_i + \delta_i$).

\subsection{Noisy Responses} \label{section:noisy_responses}

In the two previous set-ups the model deployer learns $\beta_k$ (and $W$) from the responses of the negatively classified individuals, who have perfect knowledge of their own preferences $\beta_k$ and the true SCM $\mathcal{M}$. However, in reality, often responses to such questions can be noisy and it is highly unlikely that individuals actually have perfect knowledge of the SCM. We now relax these two assumptions. First, we assume that users only know the SCM up to some noise. This means that, when evaluating which of the sets of actions $A^1$ and $A^2$ they prefer, they assume that the downstream (causal) effect of each action within the set of actions is a noisy version of the true downstream effect. If we assume that this noise is drawn from a probability distribution with mean 0 then, \textit{on average}, users have perfect knowledge of the true SCM. Secondly, we assume that individuals additionally evaluate their preferences with some noise (they are not able to perfectly evaluate which of the sets of actions they would prefer perfectly given their noisy belief of the SCM as some noise is added to the decision making process).\\

To add noise to the users' knowledge of the SCM, after an intervention on $x_i$, we add noise to the other variables $x_{\neg i}$ after the prediction step in Abduction-Action-Prediction steps \citep{pearl2016causal} (see section \ref{section:scms} for an explanation of the Abduction-Action-Prediction steps). We assume that the noise is drawn from a Gaussian distribution with mean 0 and standard deviation $\sigma_F$. We denote the SCM that the users evaluate the effects of interventions with as $\tilde{\mathcal{M}}$ with associated structural equations $\tilde{\mathcal{F}}$.\\

To add noise the the users' evaluation of their own preferences, we calculate the ratio of the costs of the sets of actions $A^1$ and $A^2$ and then add some noise. Again, we assume that this noise is Gaussian with mean 0 and with standard deviation $\sigma_B$. Now, the response of the users is defined as follows.

\begin{align}
	r & = \frac{\cost(\mathbf{x}_n, A^1_{kn} | \beta_k, \tilde{\mathcal{F}})}{\cost(\mathbf{x}_n, A^2_{kn} | \beta_k, \tilde{\mathcal{F}})} + N(0, \sigma^2_B) \\ \nonumber
	y_{kn} & = \begin{cases}
		-1 & \text{if } r \leq 1 \\
		+1 & \text{if } r > 1 \\
	\end{cases}
\end{align}

With these updated noisy responses, the optimisation problem still remains the same from the point of the model deployer. However, given the added noise, the responses of users may not always reflect the true SCM and true preferences of the users. Therefore, the role of regularisation becomes more important.

\section{Cost Function}

The cost function of actions $A$ and user preferences over feature mutability $\beta_k$ for the individual $k$ takes the below form, where $\beta \in \mathbb{R}^D$ and $D$ is the number of features\footnote{The order of the interventions in $A$ does not matter in this case, as the cost function only depends on $\delta$, not $\mathbf{x}$ (see \Cref{sequential_proposition})}. Squaring the intervention $\delta_i$ on variable $i$ has several desirable properties. Notably that (a), the cost is always greater than or equal to 0, (b) an intervention of $\delta_i=0$ results is a cost of 0 (i.e., doing nothing has no associated cost) and (c) larger interventions results in (polynomially) increasing costs, meaning that an intervention of $2a$ on variable $x_i$ has more than twice the cost of $a$.

\begin{equation}
	\texttt{cost}(A, \beta) = \sum_{i=1}^D \beta_i \delta^2_i
\end{equation}
