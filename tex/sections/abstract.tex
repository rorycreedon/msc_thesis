\clearpage
{
    \pagestyle{abstract}
    \chapter*{Abstract}
      As algorithmic decision making systems have become more widely used in the real world, interest in algorithmic recourse, a set of actions an individual can perform in order to obtain a positive outcome from a model, has also increased. This thesis addresses three key issues in the generation of algorithmic recourse. Firstly, in the real world, actions such as increasing income and increasing savings are often applied sequentially, as opposed to simultaneously. To this end, we propose a method to find the optimal \textit{sequence} of actions, making use of differentiable sorting. Secondly, we must take into account \textit{individual} preferences over how easy or difficult it is to change features. Thirdly, we must take into account the causal effects of changing one variable on other variables. To address the second and third issues, we design a novel methodology to learn both individual preferences and causal effects by presenting negatively classified individuals with a series of paired comparisons of different sequences of actions. We evaluate this methodology on synthetic data. Compared to a baseline of no knowledge over user preferences or the casual graph, we find that our methodology leads to less costly recourse.    
    \thispagestyle{abstract}
}
\newpage
