\clearpage
{
    \pagestyle{abstract}
    \chapter*{Abstract}
    \begin{itemize}
    	\item Most works in algorithmic recourse assume a simple, pre-specified cost function for changing feature values.
    	\item Understanding \textit{individual} cost functions is important for generating recourse and understanding \textit{global} cost functions is important for strategic classification.
    	\item Whilst there has been research into generating individual recourse through preference elicitation, there has not been research into learning \textit{global} cost functions.
    	\item Learning algorithms are proposed to learn cost function from the users' revealed preferences - their responses to a series of pairwise comparisons of different recourse options.
    	\item The algorithms are evaluated on synthetic and semi-synthetic data.
    	\item Recourse costs are compared for users with different protected attributes, showing if learning costs functions aids or exacerbates fairness of recourse.
    \end{itemize}
    \thispagestyle{abstract}
}
\newpage

%\clearpage
%{
%    \pagestyle{is}
%    \chapter*{Impact Statement}
%    %% Insert the Impact Statement here. Delete the following line before compiling.
%    Max 500 words. \newline\url{https://www.grad.ucl.ac.uk/essinfo/docs/Impact-Statement-Guidance-Notes-for-Research-Students-and-Supervisors.pdf} has further info. 
%    \thispagestyle{is}
%}
%\newpage

%\clearpage
%{
%    \pagestyle{dedication}
%    \chapter*{Dedication}
%    %% Insert the Dedication here.
%    \thispagestyle{dedication}
%}
%\newpage

%\clearpage
%{
%    \pagestyle{acknowledgements}
%    \chapter*{Acknowledgements}
%    %% Insert the Acknowledgements here.
%    \thispagestyle{acknowledgements}
%}
%\newpage
